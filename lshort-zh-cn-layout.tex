\newif\ifoddsidelayout

\expandafter\ifx\csname ver@layout.sty\endcsname\relax
\typeout{Package `layout' not loaded. Command \string\layoutpicture{} will have no effect.}
\def\layoutpicture{\@ifstar\@empty\@empty}
\expandafter\endinput
\fi

\newcommand\lay@layoutpic{%
    \if@twoside
    \ifoddsidelayout
    \ref@marginwidth=\cnt@oddsidemargin
    \ref@marginpar=\oneinch
    \advance\ref@marginpar by \ref@hoffset
    \advance\ref@marginpar by \cnt@oddsidemargin
    \ref@margin\ref@marginpar
    \if@reversemargin
    \advance\ref@marginpar by -\cnt@marginparsep
    \advance\ref@marginpar by -\cnt@marginparwidth
    \else
    \advance\ref@marginpar by \cnt@textwidth
    \advance\ref@marginpar by \cnt@marginparsep
    \fi
    \else
    \ref@marginwidth=\cnt@evensidemargin
    \ref@marginpar=\oneinch
    \advance\ref@marginpar by \ref@hoffset
    \advance\ref@marginpar by \cnt@evensidemargin
    \ref@margin\ref@marginpar
    \if@reversemargin
    \advance\ref@marginpar by \cnt@textwidth
    \advance\ref@marginpar by \cnt@marginparsep
    \else
    \advance\ref@marginpar by -\cnt@marginparsep
    \advance\ref@marginpar by -\cnt@marginparwidth
    \fi
    \fi
    \else
    \ref@marginwidth=\cnt@oddsidemargin
    \ref@marginpar=\oneinch
    \advance\ref@marginpar by \ref@hoffset
    \advance\ref@marginpar by \cnt@oddsidemargin
    \ref@margin\ref@marginpar
    \if@reversemargin
    \advance\ref@marginpar by -\cnt@marginparsep
    \advance\ref@marginpar by -\cnt@marginparwidth
    \else
    \advance\ref@marginpar by \cnt@textwidth
    \advance\ref@marginpar by \cnt@marginparsep
    \fi
    \fi
    \setlength{\unitlength}{.5pt}
    \begin{picture}(\cnt@paperwidth,\cnt@paperheight)
        \centering
        \thicklines
        \put(0,0){\framebox(\cnt@paperwidth,\cnt@paperheight){\mbox{}}}
        \put(0,\cnt@voffset){\dashbox{10}(\cnt@paperwidth,0){\mbox{}}}
        \put(\cnt@hoffset,0){\dashbox{10}(0,\cnt@paperheight){\mbox{}}}
        \put(\ref@margin,\ref@head){%
            \begingroup\color{lightgray}\framebox(\cnt@textwidth,\cnt@headheight)%
            {\tiny\sffamily\Headertext}\endgroup}
        \put(\ref@margin,\ref@body){%
            \framebox(\cnt@textwidth,\cnt@textheight){\sffamily\Bodytext}}
        \put(\ref@margin,\ref@foot){%
            \begingroup\color{lightgray}\framebox(\cnt@textwidth,\fheight){\tiny\sffamily\Footertext}\endgroup}
        \put(\ref@marginpar,\ref@body){%
            \begingroup\color{lightgray}\framebox(\cnt@marginparwidth,\cnt@textheight)%
            {\small\sffamily\shortstack{\MarginNotestext}}\endgroup}
        \thinlines
        \SetToHalf\PositionX\cnt@textwidth
        \advance\PositionX by \ref@margin
        \PositionY = \ref@body
        \advance\PositionY by 50
        \Identify{8}
        \InsideHArrow\cnt@textwidth
        \SetToHalf\PositionY\cnt@textheight
        \advance\PositionY by \ref@body
        \PositionX = \cnt@textwidth
        \divide\PositionX by 5
        \multiply \PositionX by 4
        \advance\PositionX by \ref@margin
        \Identify{7}
        \InsideVArrow\cnt@textheight
        \PositionY = 50
        \SetToHalf\PositionX\cnt@hoffset
        \Identify{1}
        \InsideHArrow\cnt@hoffset
        \SetToQuart\PositionY\cnt@textheight
        \advance\PositionY by \ref@body
        \ifnum\ref@marginwidth > 0
        \OutsideHArrow\ref@margin\ref@marginwidth{20}
        \PositionX = \cnt@hoffset
        \else
        \OutsideHArrow\cnt@hoffset{-\ref@marginwidth}{20}
        \PositionX = \ref@margin
        \fi
        \advance\PositionX by -30
        \Identify{3}
        \SetToQuart\PositionY\cnt@textheight
        \advance\PositionY by \ref@body
        \advance\PositionY by 30
        \SetToHalf\PositionX\cnt@marginparwidth
        \advance\PositionX by \ref@marginpar
        \Identify{10}
        \InsideHArrow\cnt@marginparwidth
        \advance\PositionY by 30
        \if@twoside
        \if@reversemargin
        \ifoddsidelayout
        \OutsideHArrow\ref@margin\cnt@marginparsep{20}
        \PositionX = \ref@margin
        \else
        \OutsideHArrow\ref@marginpar\cnt@marginparsep{20}
        \PositionX = \ref@marginpar
        \fi
        \else
        \ifoddsidelayout
        \OutsideHArrow\ref@marginpar\cnt@marginparsep{20}
        \PositionX = \ref@marginpar
        \else
        \OutsideHArrow\ref@margin\cnt@marginparsep{20}
        \PositionX = \ref@margin
        \fi
        \fi
        \else
        \if@reversemargin
        \OutsideHArrow\ref@margin\cnt@marginparsep{20}
        \PositionX = \ref@margin
        \else
        \OutsideHArrow\ref@marginpar\cnt@marginparsep{20}
        \PositionX = \ref@marginpar
        \fi
        \fi
        \advance\PositionX by -\cnt@marginparsep
        \advance\PositionX by -30
        \Identify{9}
        \PositionX = \cnt@textwidth
        \divide\PositionX by 8
        \advance\PositionX by \ref@margin
        \OutsideVArrow\ref@foot\cnt@footskip{20}{20}
        \PositionY = \ref@foot
        \advance\PositionY by \cnt@footskip
        \advance\PositionY by 30
        \Identify{11}
        \PositionX = \cnt@paperwidth
        \advance\PositionX by -50
        \PositionY = \cnt@paperheight
        \ExtraYPos = \PositionY
        \advance\ExtraYPos by -\cnt@voffset
        \advance\PositionY by \cnt@voffset
        \divide\PositionY by \tw@
        \Identify{2}
        \InsideVArrow\ExtraYPos
        \Interval = \cnt@textwidth
        \divide\Interval by 8
        \PositionX = \ref@margin
        \advance\PositionX by \Interval
        \ifnum\cnt@topmargin > \z@
        \ExtraYPos = \ref@head
        \advance\ExtraYPos\cnt@headheight
        \OutsideVArrow\ExtraYPos\cnt@topmargin{20}{20}
        \PositionY = \ExtraYPos
        \advance\PositionY by \cnt@topmargin
        \else
        \ExtraYPos = \cnt@voffset
        \OutsideVArrow\ExtraYPos{-\cnt@topmargin}{20}{20}
        \PositionY = \ExtraYPos
        \advance\PositionY by -\cnt@topmargin
        \fi
        \advance\PositionY by 30
        \Identify{4}
        \advance\PositionX by \Interval
        \OutsideVArrow\ref@head\cnt@headheight{20}{20}
        \PositionY = \ref@head
        \advance\PositionY by \cnt@headheight
        \advance\PositionY by 30
        \Identify{5}
        \advance\PositionX by \Interval
        \ExtraYPos=\ref@body
        \advance\ExtraYPos\cnt@textheight
        \OutsideVArrow\ExtraYPos\cnt@headsep{20}{20}
        \PositionY = \ref@body
        \advance\PositionY by \cnt@textheight
        \advance\PositionY by -30
        \Identify{6}
    \end{picture}
    \vskip\medskipamount
    \begingroup\footnotesize\ttfamily
    \begin{tabular}{@{}rl@{\hspace{20pt}}rl}
        1 & \oneinchtext\ + \LayOutbs\texttt{hoffset}
        & 2 & \oneinchtext\ + \LayOutbs\texttt{voffset} \\
        3 & \if@twoside
        \ifoddsidelayout \Show{cnt}{oddsidemargin}
        \else \Show{cnt}{evensidemargin}
        \fi
        \else
        \Show{cnt}{oddsidemargin}
        \fi & 4 & \Show{cnt}{topmargin} \\
        5 & \Show{cnt}{headheight}   & 6  & \Show{cnt}{headsep}        \\
        7 & \Show{cnt}{textheight}   & 8  & \Show{cnt}{textwidth}      \\
        9 & \Show{cnt}{marginparsep} & 10 & \Show{cnt}{marginparwidth} \\
        11 & \Show{cnt}{footskip} & & \Show{cnt}{marginparpush}
        \rlap{(\notshown)} \\
        & \Show{ref}{hoffset}      &    & \Show{ref}{voffset}        \\
        & \Show{cnt}{paperwidth}   &    & \Show{cnt}{paperheight}    \\
    \end{tabular}
    \endgroup
}
\def\layoutpicture{%
    \@ifstar{\lay@getvalues\lay@layoutpic}{\lay@layoutpic}}
\endinput
