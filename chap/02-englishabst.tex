%    英文摘要
\begin{englishabstract}{K-shortest path, Time-space network modeling, Time series analysis, Algorithm design, Program implementation}
    For travelers, the path with the least cost is often chosen from the origin to the destination. The shortest path problem is the basic problem in the network flow problem, which refers to choosing a path with the least cost from the starting point to the ending point. If the traveler is not only required to choose the shortest path, but also has some special needs, that is, with some specific constraints, such as having to pass a certain school, or avoid certain road sections, the problem turns into a constraint the shortest path problem. For different needs of different drivers or different needs of the same driver at different times and occasions, either the shortest distance, the shortest travel time, or the highest travel reliability are required; or for service providers, such as Baidu Maps, etc., in order to provide better services, provide travelers with multiple shortest path options, so there are multiple shortest paths (K Shortest Paths, KSP).

    Analysis of multiple shortest path problems, on the one hand, it can recommend personalized routes according to the characteristics of drivers, and can also meet the differentiated decisions made due to the temporary special needs of drivers, and can also adapt to road section failures caused by road construction or road repair. It provides a secondary short circuit for recommendation, and has a certain system fault tolerance and automatic repair ability. Providing drivers with a highly reliable and robust route navigation system in a dynamic and uncertain traffic network can greatly improve the user experience.

    This paper first studies the differences between general physical networks, stochastic networks and dynamic traffic networks, and finds that the shortest path in a general physical network is not the same as the shortest path in a traffic network where the impedance of the actual road segment is changing. The shortest path problem is studied in the dynamic time-space network to obtain better optimal route recommendation and better traffic simulation effect. The time-space network model in the form of node expansion has a stronger ability to describe various traffic behaviors such as subway transfer and intersection turning; Discretization processing, so as to realize the transformation from the k shortest path problem of the time-space network to the shortest path problem of the source point sequence to the sink point sequence in the static network.

    Then completed the design of the single-source shortest path algorithm, the k shortest path algorithm design and the k shortest path algorithm design in the time-space network, and compared the execution efficiency, advantages and disadvantages of various shortest path algorithms and usage scenarios. According to the actual problems in the transportation field The characteristic of , realizes the k-shortest path pathfinding program that solves the acyclic, and provides a basic tool for the further design and research of k-shortest path algorithm in the future.

    Finally, a simple test network is used to test the correctness of the k shortest path pathfinding program, and through the 24-node-76 road section network, the 50-node-1600-section network, the 730-node-5800-section network, and the 800-node network -8800 road section network, 1000 node -11000 road section network to test the efficiency of the k shortest path pathfinding program, the test results show that the realized k shortest path pathfinding program can complete the pathfinding task with extremely high efficiency, and can be used in large In the large-scale road network, the efficiency of the online real-time route recommendation system can also be guaranteed. After ensuring efficiency, it was used in a large-scale simulation of the urban road network in Queens, New York City.

    The research results show that: in a super large-scale road network, heuristic algorithms such as the A-star algorithm are more efficient, but there are loops in the shortest path found, but in the application scenario of traffic pathfinding, it will not. There is a situation in which a traveler makes a detour to find the shortest path, and the operating efficiency of the heuristic algorithm is related to the heuristic rules. At the same time, the found path may not be an exact optimal solution but an approximate solution. The k-shortest path obtained according to the deviated path idea is efficient enough to meet the requirements of the online real-time route recommendation and navigation system for medium-sized cities.
\end{englishabstract}