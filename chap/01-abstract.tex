%    中文摘要
%      1. 括弧内是关键字
%      2. 摘要内容
\begin{abstract}{k最短路,时间空间网络建模,时间序列分析,算法设计, 程序实现}
    对于出行者来说,从出发地到目的地往往会选择成本最小的路径。最短路问题是网络流问题中的基础问题,是指从起点到终点之间选择一条成本最小的路径。
    如果对于出行者来说,不仅仅要求选择最短路径,而且具有一些特殊的需求,即附加一些特定的限制条件,例如必须经过某所学校,或避开某些路段,则该问题转变为带有约束的最短路问题。
    对于不同驾驶者的不同需求差异或同一驾驶者不同时间场合下的不同需求差异,或要求距离最短,或要求行程时间最短,或要求行程可靠性最高;
    又或者是对于服务提供商,例如,百度地图等,为了提供更好的服务,为出行者提供多条最短路的选择方案,因此出现了多条最短路问题(K Shortest Paths, KSP)。

    多条最短路问题分析,一方面可以根据驾驶者的特点进行个性化的路线推荐,也能满足因驾驶者临时的特殊需求作出的差异性决策,还能适应因路段施工或路段修复造成路段故障时提供次短路进行推荐,
    具有一定的系统容错能力和自动修复能力。在具有动态性和不确定性的交通路网中为驾驶者提供具有高可靠性,高鲁棒性的路径导航系统能极大地改善用户体验。

    本文首先研究了一般物理网络,随机网络和动态的交通网络中之间的差异,发现在一般物理网络中的最短路径与实际路段阻抗不断变化的交通网络中的最短路径并不相同,因此需要在动态的时间空间网络中对最短路问题进行研究从而获得更好的最优路径推荐和更好的交通仿真效果;
    基于拆点的思想完成将节点阻抗和路段阻抗在形式表达上的统一,提出了节点拓展形式的时间空间网络模型,对地铁换乘、交叉口转向等各种交通行为的描述能力更强;
    根据交通流微观上连续,宏观上离散且周期性变化的特点,对时间空间网络进行离散化处理,从而实现从时间空间网络k最短路问题到
    静态网络中源点序列到汇点序列最短路问题的转化。

    随后从Dijkstra算法出发,完成了单源最短路算法的设计,并借助偏离路径的思想完成了k最短路算法的设计。在静态k最短路算法设计的基础上,结合交通流自身的特性,提出两个假设条件,
    在此基础上完成对时间空间网络中的k最短路算法设计。接着对比了各种最短路算法的执行效率和优劣势以及使用场景,多条最短路问题相较于最短路问题提供了更多的可选方案,最短路算法仅仅考虑了将出行者的需求局限在路径最短的条件之下,
    既不能满足其它需求用户的需要,也不利于改善服务提供商的提供的服务。
    同时当路网系统某一结点或路段发生故障时,最短路径可能恰好失效,需要重新计算最短路。对于大规模路网中计算所需的时间是不可忽略的,会造成极其糟糕的用户体验。
    多条最短路作为最短路问题的拓展,既满足了出行者差异化的需要, 又为路网系统提供了高可靠性和高容错性,还使得服务提供商能为用户提供更好的服务。

    之后,根据交通领域实际问题的特点,通过偏离路径的思想,实现了求解无环的k最短路寻路程序,
    同时提供一个与多条最短路算法设计密切相关的实际应用案例-路径导航系统,通过分别对其进行手工求解和程序求解来验证算法设计的正确性和程序设计的正确性和可靠性。
    并且在大规模路网中,手工求解方式变得不可行,因此程序求解是作为解决最短路问题的唯一方式,该程序的设计为将来进一步的k最短路算法设计研究提供了基础工具。

    最后通过一个简单的测试网络对k最短路寻路程序的正确性进行检测,并分别通过24结点-76路段网络,50结点-1600路段网络,730结点-5800路段网络,800结点-8800路段网络,1000结点-11000路段网络对k最短路寻路程序进行效率测试,
    测试结果表明实现的k最短路寻路程序可以以极高的效率完成寻路任务,既可以使用于大规模的路网中,也能保证在线实时路径推荐系统的效率。确保效率之后将其用于大规模的纽约市皇后区的城市道路网络模拟中。
    
    研究结果表明: 在超大规模的路网,启发式算法如A-star算法等的运行效率更高,但是寻找到的最短路径中存在环路,而在交通寻路的应用场景之下,不会存在出行者绕行一圈寻找最短路的情况,且启发式算法的运行效率和启发式规则有关,同时寻找到的路径有可能不是精确的最优解而是近似解。
    而根据偏离路径思想求得的k最短路在效率上足够满足中等城市规模的在线实时路径推荐导航系统的要求。
\end{abstract}