\appendix


\chapter{主程序}\label{ch:主程序}
\begin{lstlisting}[label={lst:lstlisting}]
#include <limits>
#include <set>
#include <map>
#include <queue>
#include <string>
#include <vector>
#include <fstream>
#include <iostream>
#include <algorithm>
#include "GraphElements.h"
#include "Graph.h"
#include "DijkstraShortestPathAlg.h"
#include "YenTopKShortestPathsAlg.h"
using namespace std;

void testDijkstraGraph() {
    Graph *my_graph_pt = new Graph("C:\\Users\\24563\\Desktop\\新建文件夹\\k-shortest-paths-cpp-version\\data\\test_15");
    DijkstraShortestPathAlg shortest_path_alg(my_graph_pt);
    BasePath *result = shortest_path_alg.get_shortest_path(my_graph_pt->get_vertex(0), my_graph_pt->get_vertex(14));
    result->PrintOut(cout);
}

void testYenAlg(int source, int sink, int k, string filename) {
    Graph my_graph(filename);

    YenTopKShortestPathsAlg yenAlg(my_graph, my_graph.get_vertex(source), my_graph.get_vertex(sink));


    int i = 0;
    while (yenAlg.has_next()) {
        ++i;
        cout << "从结点" << source << "到结点" << sink << "的第" << i << "最短路为:" << endl;
        yenAlg.next()->PrintOut(cout);
        if (i == k) {
            break;
        }
    }
}

int main(...) {
    cout << "请输入待求k最短路的起点和终点以及k值" << endl;
    int source;
    int sink;
    int k;
    cin >> source >> sink >> k;
    cout << "请输入数据文件所在的文件路径" << endl;
    string filepath = "C:\\Users\\24563\\Desktop\\新建文件夹\\k-shortest-paths-cpp-version\\data\\test_1";
    cin >> filepath;
    testYenAlg(source, sink, k, filepath);
    int flag;
    cout << "输入任意键退出" << endl;
    cin >> flag;
    return 0;
}


\end{lstlisting}


\chapter{K最短路实现}\label{ch:k最短路实现}
\begin{lstlisting}[
    language=C++,
    basicstyle=\ttfamily,
    breaklines=true,
    label={lst:lstlisting2}]
#include <set>
#include <map>
#include <queue>
#include <vector>
#include "GraphElements.h"
#include "Graph.h"
#include "DijkstraShortestPathAlg.h"
#include "YenTopKShortestPathsAlg.h"

using namespace std;

void YenTopKShortestPathsAlg::clear() {
    m_nGeneratedPathNum = 0;
    m_mpDerivationVertexIndex.clear();
    m_vResultList.clear();
    m_quPathCandidates.clear();
}

void YenTopKShortestPathsAlg::_init() {
    clear();
    if (m_pSourceVertex != NULL && m_pTargetVertex != NULL) {
        BasePath *pShortestPath = get_shortest_path(m_pSourceVertex, m_pTargetVertex);
        if (pShortestPath != NULL && pShortestPath->length() > 1) {
            m_quPathCandidates.insert(pShortestPath);
            m_mpDerivationVertexIndex[pShortestPath] = m_pSourceVertex;
        }
    }
}

BasePath *YenTopKShortestPathsAlg::get_shortest_path(BaseVertex *pSource, BaseVertex *pTarget) {
    DijkstraShortestPathAlg dijkstra_alg(m_pGraph);
    return dijkstra_alg.get_shortest_path(pSource, pTarget);
}

bool YenTopKShortestPathsAlg::has_next() {
    return !m_quPathCandidates.empty();
}

BasePath *YenTopKShortestPathsAlg::next() {
    //1. Prepare for removing vertices and arcs
    BasePath *cur_path = *(m_quPathCandidates.begin());//m_quPathCandidates.top();

    //m_quPathCandidates.pop();
    m_quPathCandidates.erase(m_quPathCandidates.begin());
    m_vResultList.push_back(cur_path);

    int count = m_vResultList.size();

    BaseVertex *cur_derivation_pt = m_mpDerivationVertexIndex.find(cur_path)->second;
    vector<BaseVertex *> sub_path_of_derivation_pt;
    cur_path->SubPath(sub_path_of_derivation_pt, cur_derivation_pt);
    int sub_path_length = sub_path_of_derivation_pt.size();

    //2. Remove the vertices and arcs in the graph
    for (int i = 0; i < count - 1; ++i) {
        BasePath *cur_result_path = m_vResultList.at(i);
        vector<BaseVertex *> cur_result_sub_path_of_derivation_pt;

        if (!cur_result_path->SubPath(cur_result_sub_path_of_derivation_pt, cur_derivation_pt)) continue;

        if (sub_path_length != cur_result_sub_path_of_derivation_pt.size()) continue;

        bool is_equal = true;
        for (int i = 0; i < sub_path_length; ++i) {
            if (sub_path_of_derivation_pt.at(i) != cur_result_sub_path_of_derivation_pt.at(i)) {
                is_equal = false;
                break;
            }
        }
        if (!is_equal) continue;

        //
        BaseVertex *cur_succ_vertex = cur_result_path->GetVertex(sub_path_length + 1);
        m_pGraph->remove_edge(make_pair(cur_derivation_pt->getID(), cur_succ_vertex->getID()));
    }

    //2.1 remove vertices and edges along the current result
    int path_length = cur_path->length();
    for (int i = 0; i < path_length - 1; ++i) {
        m_pGraph->remove_vertex(cur_path->GetVertex(i)->getID());
        m_pGraph->remove_edge(make_pair(cur_path->GetVertex(i)->getID(), cur_path->GetVertex(i + 1)->getID()));
    }

    //3. Calculate the shortest tree rooted at target vertex in the graph
    DijkstraShortestPathAlg reverse_tree(m_pGraph);
    reverse_tree.get_shortest_path_flower(m_pTargetVertex);

    //4. Recover the deleted vertices and update the cost and identify the new candidates results
    bool is_done = false;
    for (int i = path_length - 2; i >= 0 && !is_done; --i) {
        //4.1 Get the vertex to be recovered
        BaseVertex *cur_recover_vertex = cur_path->GetVertex(i);
        m_pGraph->recover_removed_vertex(cur_recover_vertex->getID());

        //4.2 Check if we should stop continuing in the next iteration
        if (cur_recover_vertex->getID() == cur_derivation_pt->getID()) {
            is_done = true;
        }

        //4.3 Calculate cost using forward star form
        BasePath *sub_path = reverse_tree.update_cost_forward(cur_recover_vertex);

        //4.4 Get one candidate result if possible
        if (sub_path != NULL) {
            ++m_nGeneratedPathNum;

            //4.4.1 Get the prefix from the concerned path
            double cost = 0;
            reverse_tree.correct_cost_backward(cur_recover_vertex);

            vector<BaseVertex *> pre_path_list;
            for (int j = 0; j < path_length; ++j) {
                BaseVertex *cur_vertex = cur_path->GetVertex(j);
                if (cur_vertex->getID() == cur_recover_vertex->getID()) {
                    //j = path_length;
                    break;
                } else {
                    cost += m_pGraph->get_original_edge_weight(cur_path->GetVertex(j), cur_path->GetVertex(1 + j));
                    pre_path_list.push_back(cur_vertex);
                }
            }
            //
            for (int j = 0; j < sub_path->length(); ++j) {
                pre_path_list.push_back(sub_path->GetVertex(j));
            }

            //4.4.2 Compose a candidate
            double weight_cost = cost + sub_path->Weight();
            delete sub_path;
            sub_path = new Path(pre_path_list, weight_cost);

            //4.4.3 Put it in the candidate pool if new
            if (m_mpDerivationVertexIndex.find(sub_path) == m_mpDerivationVertexIndex.end()) {
                m_quPathCandidates.insert(sub_path);
                m_mpDerivationVertexIndex[sub_path] = cur_recover_vertex;
            }
        }

        //4.5 Restore the edge
        BaseVertex *succ_vertex = cur_path->GetVertex(i + 1);
        m_pGraph->recover_removed_edge(make_pair(cur_recover_vertex->getID(), succ_vertex->getID()));

        //4.6 Update cost if necessary
        double cost_1 = m_pGraph->get_edge_weight(cur_recover_vertex, succ_vertex) + reverse_tree.get_start_distance_at(succ_vertex);

        if (reverse_tree.get_start_distance_at(cur_recover_vertex) > cost_1) {
            reverse_tree.set_start_distance_at(cur_recover_vertex, cost_1);
            reverse_tree.set_predecessor_vertex(cur_recover_vertex, succ_vertex);
            reverse_tree.correct_cost_backward(cur_recover_vertex);
        }
    }

    //5. Restore everything
    m_pGraph->recover_removed_edges();
    m_pGraph->recover_removed_vertices();

    return cur_path;
}

void YenTopKShortestPathsAlg::get_shortest_paths(BaseVertex *pSource, BaseVertex *pTarget, int top_k, vector<BasePath *> &result_list) {
    m_pSourceVertex = pSource;
    m_pTargetVertex = pTarget;

    _init();
    int count = 0;
    while (has_next() && count < top_k) {
        next();
        ++count;
    }

    result_list.assign(m_vResultList.begin(), m_vResultList.end());
}
\end{lstlisting}


\chapter{时间拓展图转化为静态网络}\label{ch:时间拓展图转化为静态网络}
\begin{lstlisting}[
    language=C++,
    basicstyle=\ttfamily,
    breaklines=true,
    label={lst:lstlisting4}]

struct Edge {
int to;
vector<int> edgeWeight;
explicit Edge(int to) {
this->to = to;
}
};
void staticGraph() {
int vertexCounts;
int edgeCounts;
cout << "请输入图中的顶点数和边数: ";
cin >> vertexCounts >> edgeCounts;
vector<vector<Edge>> graph(vertexCounts);
int from, to;
int weight;
while (edgeCounts--) {
cin >> from >> to;
graph[from].emplace_back(to);
int i = 0;
while (i++ < seriesCount) {
cin >> weight;
graph[from].back()
.edgeWeight.push_back(weight);
}
}
int exVertexCounts = vertexCounts * seriesCount;
vector<vector<int>>matrix(exVertexCounts, vector<int>(exVertexCounts, -1));
for (int i = 0; i < graph.size(); i++) {
for (int j = 0; j < graph[i].size(); ++j) {
int from = i;
int to = graph[i][j].to;
for (int k = 0; k < graph[i][j].edgeWeight.size(); ++k) {
int weight = graph[i][j].edgeWeight[k];
matrix[from * seriesCount + k][to * seriesCount + (k + weight) % seriesCount] = weight;
}
}
}
for (int i = 0; i < graph.size(); i++) {
for (const auto &item : graph[i]) {
cout << i << "->" << item.to << ":";
for (int cost : item.edgeWeight) {
cout << cost << " ";
}
cout << endl;
}
}
for (auto &i : matrix) {
for (int j : i) {
cout << j << " ";
}
cout << endl;
}
}
\end{lstlisting}


\chapter{路网模拟实验代码}\label{ch:路网模拟实验代码}
\begin{lstlisting}[
    language=Python,
    morekeywords={as},
    label={lst:lstlisting8}]

import osmnx as ox
import pandas as pd
import networkx as nx
import numpy as np

G = ox.graph_from_place('Queens, New York, USA', network_type='drive')
remove_list = []
num_nodes = len(G.nodes)
for node in G.nodes:
reach = len(nx.descendants(G, node))
if reach < num_nodes / 10:
remove_list.append(node)
for node in remove_list:
G.remove_node(node)
G = nx.convert_node_labels_to_integers(G, label_attribute='old_node_ID')
ox.plot_graph(G)

G = ox.add_edge_speeds(G)
G = ox.speed.add_edge_travel_times(G, precision=1)

speed_df = pd.read_csv('../data/streetInfo.csv')
speed_df = speed_df[['osm_way_id', 'hour', 'speed']]
speed_df.head()

speed_dict = dict([((t.osm_way_id, t.hour),t.speed) for t in speed_df.itertuples()])
speed_dict[(5029221, 12)]

for edge in G.edges:
    edge_obj = G[edge[0]][edge[1]][edge[2]]
    wayid = edge_obj['osmid']
try:
    speed = speed_dict[wayid] * 1.60934
    distance = edge_obj['length'] / 1000
    travel_time = distance / speed * 60
except:
    travel_time = edge_obj['travel_time'] / 60
    G[edge[0]][edge[1]][edge[2]]['um_travel_time'] = travel_time
# %%
num_nodes = len(G.nodes)
shortest_time = np.zeros((num_nodes, num_nodes))
path_generator = nx.shortest_path_length(G, weight='um_travel_time')
for origin_data in path_generator:
    origin = origin_data[0]
    dist_dict = origin_data[1]
    for destination in dist_dict:
        shortest_time[origin, destination] = dist_dict[destination]
\end{lstlisting}


%\newtheorem{theorem}{定理}
%\begin{theorem}
%    欧几里得第二定理(素数有无穷多个)\\
%    证明:用反证法。假设素数有有限个($N$个),记为$p_1,p_2,\dots,p_N$。则我们构造一个新的数,
%    \[
%        n=p_1p_2\dots p_N+1.
%    \]
%    由于$p_i,i=1,2,\dots,N$为素数,则一定不为$1$。于是对于任意的$p_i,i=1,2,\dots, N$,有
%    \[
%        p_i\not|n
%    \]
%    这表明,要么$n$本身为素数,要么$n$为合数,但是存在$p_1,p_2,\dots,p_N$之外的其他素数能够将$n$进行素因子分解。
%    不管哪种情况,都表明存在更多的素数。定理得证。\qed
%\end{theorem}
%
%
%\chapter{$\sqrt{2}$是无理数的证明}
%\begin{theorem}
%    $\sqrt{2}$是无理数。\\
%    证明:用反证法。假设$\sqrt{2}$是有理数,则可表示为两个整数的商,即$\exists p,q, q\ne0$
%    \[
%        \sqrt{2}=\frac{p}{q}
%    \]
%    不失一般性,我们假设$p,q$是既约的,即$\gcd(p,q)=1$。对上式两边平方可得\\
%    \begin{align*}
%        2& =\frac{p^2}{q^2}\\
%        p^2&=2q^2.
%    \end{align*}
%    表明$p^2$为偶数,因此$p$为偶数,记$p=2m$。则
%    \begin{align*}
%        p^2&=4m^2=2q^2\\
%        q^2&=2m^2.
%    \end{align*}
%    表明$q$也为偶数,因此它们有公共因子$2$。这与它们既约的假设矛盾。定理得证。\qed
%\end{theorem}