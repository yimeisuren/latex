\let\cleardoublepage\clearpage


\chapter{最短路理论基础}\label{ch:最短路理论基础}
最短路问题与我们的日常生活息息相关,例如我们常常需要使用某种地图软件或者导航系统来获取从一个地方到另一个地方的路径。我们便可以得到对应该问题的图模型:以顶点对应交叉口,以边对应公路,边的权重对应经过该路段的成本,可以是时间、距离、偏好等等。


\section{符号定义}\label{sec:符号定义}
%center居中环境
\begin{center}
    %    tabular表格
    \begin{tabular}{l l}%    c表示居中, l表示左对齐, r表示右对齐
        \hline
        符号         & 含义                                       \\
        \hline%hline表示水平分割线, 使用|表示垂直分割线
        $G(V,E,W)$ & 网络                                       \\
        $N$        & 图中顶点的数量                                  \\
        $M$        & 图中边(弧)的数量                                \\
        $s$        & 最短路的源点                                   \\
        $t$        & 最短路的汇点                                   \\
        $D(u)$     & 从源点$s$到节点$u$的实际最短路距离                     \\
        $dis(u)$   & 从源点$s$到节点$u$的预测最短路距离                     \\
        $w(u,v)$   & 边的权重                                     \\
        $S$        & 已确定最短路径长度的节点的集合                          \\
        $T$        & 未确定最短路径长度的节点的集合                          \\
        $(i)$      & 表示图中的第$i$个节点,$i$的范围从$1$到$n$              \\
        $d_{ij}$   & 表示节点(i)到节点(j)之间的路段距离                     \\
        $A^k$      & 表示从$(1)$到$(n)$的第$k$条最短路径                 \\
        $A^k_i$    & 表示第$k-1$条最短路$A^{k-1}$上的第$i$个结点为偏离点时的偏离路径 \\
        $(R^k)_i$  & 源点到偏离点的距离长度                              \\
        $S^k_i$    & 偏离点到汇点的距离长度                              \\
        \hline
    \end{tabular}
\end{center}


\section{相关概念}\label{sec:相关概念}
\begin{itemize}%无序列表
    \item \textbf{距离约束条件}\\对于网络中的任意一条路径,均满足$dis(u)\geq D(u)$,当且仅当当前路径对应实际的最短路时等号成立
    \item \textbf{偏离点}\\偏离点表示最短路径和偏离路径$A^k$和$A^{k-1}$的前$i-1$个结点相同,在第$i$个结点处发生偏离, 其中第$k$条最短路径中的第$i$个结点到第$i+1$个结点的边, 即$(i)^k \to (i+1)^k$不能和现有的前$k-1$条最短路$A^j$中的边重复,其中$j$的范围从$1$到$k-1$
    \item \textbf{偏离路径}\\表示第$k-1$条最短路$A^{k-1}$上的第$i$个结点为偏离点时的偏离路径,$i$的范围从$1$到$Q_k$,其中$Q_k$表示第$k-1$条最短路径中的结点数量-1(不包括汇点),且不能在汇点处偏离。
    \item \textbf{根路径}\\表示第$k$条最短路径和第$k-1$条最短路径在结点$i$之前相同的路径
    \item \textbf{偏离距离}\\表示偏离路径$A^k$中第$i$个结点到汇点的距离,即$(i)\to\dots\to(n)$的距离
    \item \textbf{偏离路径距离}\\偏离距离与根距离之和
    \item \textbf{邻接矩阵}\\邻接矩阵是表示顶点之间相邻关系的矩阵。对于无向图而言,邻接矩阵是一个对称图,根据路径权重,一般将主对角线上的元素的路径阻抗设置为-1,表示不存在自环的回路。对于副对角线上的元素则不能唯一确定,且由于无向图中的邻接矩阵是对称矩阵,所以对于大规模的邻接矩阵会存在大量的冗余数据,因此可以使用压缩存储的方法来节省存储空间,提高存储效率和检索效率。对于有n个节点的图,如果使用邻接矩阵存储,则需要$n^2$的存储空间。
    \item \textbf{邻接表}\\邻接表是图论中用于描述一种相邻关系的数据结构。邻接表这种数据结构的优点是存储效率高,对于相邻边的检索效率高,可以使用出边表和入边表快速查找到某个节点的全部信息。邻接表美中不足的地方在于无法明确在什么地方保存相关边的长度或保存该长度所需要的额外代价。在一个邻接表中,给定任意一个顶点,可以在O(1)的时间内找到该顶点的出边和入边,而在邻接矩阵中需要花费O(n)的时间,而在最短路径算法中常常需要使用到出边和入边,因此使用邻接表对于实现最短路径算法的效率会更高。
    \item \textbf{稀疏图}\\如果网络中路径的个数远远小于其节点的平方,则称为稀疏图,而当网络中的路径个数接近其节点数的平方时,则称为稠密图。一般情况下,稀疏图使用邻接表进行存储,稠密图选择邻接矩阵进行存储。交通网络作为一种典型的稀疏图,因此之后对交通网络的存储会选择使用邻接表的方式进行存储,而对于动态的交通网络,将会通过对邻接表的表示方法进行拓展从而完成动态交通网络的存储。
    \item \textbf{有向图}\\对于图中的每一条边(u,v),若不仅表示结点u和结点v直接相连,同时表示方向,则称为有向图。其中节点u称为边的起点,节点v称为边的终点,同时节点u称为节点v的直接前驱,节点v称为节点u的直接后继。而无向图则仅仅表示结点u和结点v的连接性,所以可以通过两条有向边u`->`v,v`->`u进行表示,因此有向图的表达能力更强,适用于两个方向交通条件存在差异的网络建模中。
    \item \textbf{邻域}\\若结点u和结点v分别是边e的起点和终点,则称结点u和结点v是相邻的,称结点u(或者结点v)和边e是关联的。结点u的领域由与结点u关联的边e的关联结点v组成的集合,记为N(u)。
    \item \textbf{简单图}\\若一个图中没有单个结点构成的自回路,同时任意两个相邻结点的某一方向上至多仅有一条边,那么这样的图称为简单图。在后面的讨论中,对网络的建模将默认为简单图。
    \item \textbf{度}\\图中和一个顶点相关联的边的数量称为顶点的度数。度可以分为出度和入度,由于对图的存储必须包括对顶点的存储和对顶点关系的存储,因此图的邻接表存储方法是一种空间利用率最高的方法。在一般的需求中,往往只需要存储结点的出度或入度,但是对于速度要求较高的场景下,也可以以空间换时间的方法来同时存储图的出度表和入度表,例如A-star算法中建立反向图时则需要使用到结点的入度表。
    \item \textbf{正权图}\\若图G(V,E)的每条边e=(u,v)都被赋予一个数值作为该条边的权重,则将该图G称为加权图。如果每条边的权重数值都是一个非负实数,则称该图为正权图,例如经典的最短路算法Dijkstra算法要求图为正权图。
    \item \textbf{路径}\\途径是指由网络中若干个节点前后连接起来的边的集合。这个边集相邻边满足前一条边的终点为后一条边的起点,因此途径可以简写为$v_0\to v_1\to \dots \to v_n$,其中n称为途径的长度。若对于一条途径中任意两条边都不相同,则称该途径为迹。对于一条迹,若其中任意两个节点都不相同,则称成路径或简单路径。在交通领域中,所求的最短路径默认为不存在两次经过重复的节点,即路径中不存在环路。
    \item \textbf{反向图}\\在原有的有向图的基础上,对于每一条表示u$\to$v路段的有向边(u,v),通过建立一条与之权重相同,方向相反的有向边(v,u)来表示v$\to$u。反向图在最短路算法中的A-star算法中有着相当重要的作用,可以通过预处理来获取到当前节点到目标节点的预计最短路代价,同时也可以执行双向搜索算法,从而加快最短路算法的搜索效率。
\end{itemize}


\section{松弛操作}\label{sec:松弛操作}
对于某条特定的边(u,v),可以通过公式进行松弛操作:
\begin{equation}
    dis(v) = \min(dis(v),dis(u)+w(u,v))\label{eq:equation}
\end{equation}
松弛操作是一种典型的动态规划,其具体含义为:对于任意一条边,其状态只有两种情况,(a)处于最短路径中,(b)不处于最短路径中;
如果边(u,v)处于最短路径中,则表明从源点s到节点v的最短距离必定等于从源点s到节点u的最短距离加上边(u,v)的权值,即满足
\begin{equation}
    D(v) = D(u) + w(u,v)\label{eq:equation2}
\end{equation}

松弛操作的本质是判断一个顶点是否有更好的选择路径,如果已知的最短路径已经是实际上的最短路径,则不能找到一个额外的结点来使得上述公式成立。
描述松弛一般是指对边的松弛,在实际的操作过程往往使用对顶点进行松弛操作。所谓的对顶点进行松弛,便是对一个顶点的所有出边进行松弛操作。


\section{拉格朗日乘子法}\label{sec:拉格朗日乘子法}
拉格朗日乘子法作为一种优化算法,拉格朗日乘子法主要用于解决约束优化问题,它的基本思想就是通过引入拉格朗日乘子来将含有n个变量和k个约束条件的约束优化问题转化为含有(n+k)个变量的无约束优化问题。
拉格朗日乘子背后的数学意义是其为约束方程梯度线性组合中每个向量的系数。

如何将一个含有n个变量和k个约束条件的约束优化问题转化为含有(n+k)个变量的无约束优化问题?
拉格朗日乘数法从数学意义入手,通过引入拉格朗日乘子建立极值条件,对n个变量分别求偏导对应了n个方程,然后加上k个约束条件(对应k个拉格朗日乘子)一起构成包含了(n+k)变量的(n+k)个方程的方程组问题,这样就能根据求方程组的方法对其进行求解。

拉格朗日乘子法除了可以解决等式约束外,还可以解决不等式约束。大多数情况下等式约束不足以描述实际问题,因此拉格朗日乘数法在引入KKT条件之后可以用来处理不等式约束问题。KKT条件是指在满足一些有规则的条件下, 一个非线性规划(Nonlinear Programming)问题能有最优化解法的一个必要和充分条件。
这是一个广义化拉格朗日乘数的成果。一般地, 一个最优化数学模型的列标准形式参考开头的式子, 所谓 Karush-Kuhn-Tucker 最优化条件,就是指上式的最优点$x^*$必须满足下面的条件:
\begin{equation}
    \left.
    \begin{array}  { l  }
    { L ( x , u ) = f ( x ) + \sum _ { k = 1 } ^ { a } u _ { k } g _ { k } ( x ) }
        \\ { \quad u _ { k } (x) \geq 0 }
        \\ { g _ { k } ( x ) \leq 0 }
    \end{array}
    \right.\label{eq:equation3}
\end{equation}

KKT条件是拉格朗日乘子法的泛化,再引入了等式约束和不等式约束后,可以将原问题转为化对偶问题进行求解,当满足一定条件时,对偶问题的解和原问题的解是相同的。此时可以将问题转化为
\begin{equation}
    \min_{x}\max _{\mu}L(x, \mu)= \max _{\mu}\min m_{x}L(x, \mu)= \min _{k}f(x)=f(x^{*})\label{eq:equation4}
\end{equation}