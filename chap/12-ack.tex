\let\cleardoublepage\clearpage
\acknowledgement
光阴似箭,转眼间自己的大学生活即将画上句号,心中颇多感慨。在九龙湖的这四年间,在东南大学交通学院的悉心栽培下,我从高考失利的懵懂少年蜕变为投身交通事业的青年,从学术门外汉蜕变为初具科研能力的“准研究生”。在论文搁笔之际,深感自己的成长有赖诸位提携,因此怀揣感恩的心写下这段文字。

论文的写作过程是艰苦的,但我有幸得到了各位同学、朋友、同事和亲人的教诲和帮助。没有他们,也就没有论文的最终成果。

首先,衷心感谢我的导师程琳教授。本文是在程老师的悉心指导下完成的。从本文的选题、构思、写作、修改直到最后定稿,都凝聚着导师的智慧、才华与心血。程老师学识渊博、治学严谨求实、看待问题高屋建瓴、对待工作非常负责,为人随和坦诚,他的言传身教将使我终生受益。

师恩难忘,在此,向程老师表达我最诚挚的敬意与谢意!您深厚的专业素养、务实的工作作风、淡泊的处事态度、幽默的语言风格永远值得我学习。从大二开始,您是我们班级的讲授交通工程基础的老师,带领我步入学术之门。从毕设题目的确定到具体研究工作的开展,再到论文撰写和修改,无不倾注了您的心血。您还鼓励我继续在交工领域奋斗,去做有意义的科研。

同时我还要感谢交通学院的很多老师。从我入学至今,他们在有形无形中、有意无意中给予我很多知识,给我的选题、论证、预答辩提供了诸多的启发与帮助。
感谢学院里曾指导或帮助过我的陈峻教授、任刚教授、项乔君教授、张文波副教授、曲栩副教授、施晓蒙老师、杨顺新老师等,以及关心和支持我的辅导员李莹老师、樊炜昊学长。

感谢程琳教授课题组的各位学长学姐,特别是曾具体指导我科研工作的宋茂灿,拾一师兄。每当自己为细节问题所困扰时,你们的点拨总能使我茅塞顿开。

我还要感谢我的同窗同学,他们在论文写作过程中给予了我很多的帮助。感谢好友王志宇,你永远乐意与我讨论学习和生活中的各种问题;感谢好友杨丹,总是在我需要的时候提供关系和问候;感谢好友鹿原,和你一起并肩作战度过的考研岁月是我人生中永远难忘的经历;感谢朱宇、张博然、尚香文、陈新、梁家乐、何宛锴、叶芊芊、杨帅、匡青云等同学,你们是我在学术和人生路上的灯塔和标杆,祝愿你们前程似锦。